\documentclass[letterpaper, 11pt]{article}
\usepackage{comment} % enables the use of multi-line comments (\ifx \fi) 
\usepackage[utf8]{inputenc}
\usepackage{amsmath}
\usepackage{textcomp}
\usepackage{gensymb}
\usepackage{fullpage} % changes the margin
\usepackage[letterpaper, total={7.5in, 10in}]{geometry}
\newtheorem{theorem}{Theorem}
\newtheorem{corollary}{Corollary}
\usepackage{graphicx}
\usepackage{tikz}
\usetikzlibrary{arrows}
\usepackage{verbatim}
\usepackage[numbered]{mcode}
\usepackage{float}
\usepackage{tikz}
    \usetikzlibrary{shapes,arrows}
    \usetikzlibrary{arrows,calc,positioning}

    \tikzset{
        block/.style = {draw, rectangle,
            minimum height=1cm,
            minimum width=1.5cm},
        input/.style = {coordinate,node distance=1cm},
        output/.style = {coordinate,node distance=4cm},
        arrow/.style={draw, -latex,node distance=2cm},
        pinstyle/.style = {pin edge={latex-, black,node distance=2cm}},
        sum/.style = {draw, circle, node distance=1cm},
    }
\usepackage{xcolor}
\usepackage{mdframed}
\usepackage[shortlabels]{enumitem}
\usepackage{indentfirst}
\usepackage{hyperref}
\usepackage[version=4]{mhchem}
\usepackage{xeCJK}
\setCJKmainfont{SimSun}
\usepackage{cancel}
\AtBeginDvi{\input{zhwinfonts}}
    
\renewcommand{\thesubsection}{\thesection.\alph{subsection}}

\newenvironment{problem}[2][Problem]
    { \begin{mdframed} \textbf{#1 #2} \\}
    {  \end{mdframed}}
\newenvironment{guideln}[1][Homework submission guidelines:]
    { \begin{mdframed}[backgroundcolor=yellow!80] \textbf{#1} \\}
    {  \end{mdframed}}

% Define solution environment
\newenvironment{solution}
    {\textbf{\textit{Ans.}}\\}
    { }
%%%%%%%%%%%%%%%%%%%%%%%%%%%%%%%%%%%%%%%%%%%%%%%%%%%%%%%%%%%%%%%%%%%%%%%%%%%%%%%%%%%%%%%%%%%%%%%%%%%%%%%%%%%%%%%%%%%%%%%%%%%%%%%%%%%%%%%%
\begin{document}
%Header-Make sure you update this information!!!!
\noindent
%%%%%%%%%%%%%%%%%%%%%%%%%%%%%%%%%%%%%%%%%%%%%%%%%%%%%%%%%%%%%%%%%%%%%%%%%%%%%%%%%%%%%%%%%%%%%%%%%%%%%%%%%%%%%%%%%%%%%%%%%%%%%%%%%%%%%%%%
\large\textbf{Homework - \#1} \hfill \textbf{Leonardo Sánchez} |  孫\text{ }力弘   \\
Physics III - Statics \hfill Civil \& Electromechanical Engineering \\
Professor Yao-Chung Chen | 陳\text{ }堯中 \hfill Due date: $10^{th}$ August, 2020\\
Teacher Assistant: Cesar Ayala\hfill UPTP \\
\noindent\rule{7.5in}{2.8pt}
%%%%%%%%%%%%%%%%%%%%%%%%%%%%%%%%%%%%%%%%%%%%%%%%%%%%%%%%%%%%%%%%%%%%%%%%%
% Problem 1
%%%%%%%%%%%%%%%%%%%%%%%%%%%%%%%%%%%%%%%%%%%%%%%%%%%%%%%%%%%%%%%%%%%%%%%%%%%%%%%%%%%%%%%%%%%%%%%%%%%%%%%%%%%%%%%%%%%%%%%%%%%%%%%%%%%%%%%%
\begin{problem}{1 - (Page 15 - R. C. Hibbeler $14^{th}$ ed.)}
    Evaluate each of the following and express with an appropriate prefix:
     (a) $(430 \text{ kg})^2$, (b) $(0.002\text{ mg})^2$, and
    (c) $(230\text{ m})^3$.
\end{problem}
%%%%%%%%%%%%%%%%%%%%%%%%%%%%%%%%%%%%%%%%%%%%%%%%%%%%%%%%%%%%%%%%%%%%%%%%%
\begin{solution}
(a)
    \[(430\text{ kg})^2=18\underline{4}\, 900\text{ kg}^2 = 0.18\underline{4}\, 9\times 10^6 \left(\frac{1 \text{ Mg}}{10^3 \text{ kg}}\right)^2 \text{ kg²}=\]
    \[0.18\underline{4}\,\, 9\times \cancel{10^6} \times \frac{1 \text{ Mg}^2}{\cancel{10^6} \cancel{\text{ kg²}}}\times \cancel{\text{ kg²}}=\boxed{0.18\underline{5}\text{ Mg²}}\]
\end{solution}
(b)
    \[(0.002\text{ mg})^2=0.000\,00\underline{4}\text{ mg}^2 = \underline{4}\times 10^{-6} \left(\frac{1 \,\mu\text{g}}{10^{-3} \text{ mg}}\right)^2 \text{ mg²}=\]
    \[\underline{4}\times \cancel{10^{-6}} \times \frac{1 \mu \text{g}^2}{\cancel{10^{-6}} \cancel{\text{ mg²}}}\times \cancel{\text{ mg²}}=\boxed{4.00\mu\text{ g²}}\]
(c)
    \[(230 \text{ m})^3=12\,\underline{1}67\,000\text{ m}^3 =0.012\,\underline{2}\times 10^{9} \left(\frac{1 \text{ km}}{10^{3} \text{ m}}\right)^3 \text{ m³}=\]
    \[0.012\,\underline{2}\times \cancel{10^{9}} \times \frac{1 \text{ km}^3}{\cancel{10^{9}} \cancel{\text{ mg²}}}\times \cancel{\text{ m³}}=\boxed{0.0122\text{ km³}}\text{ or }\boxed{12.2 \text{Mm³}}\]
% Problem 2
%%%%%%%%%%%%%%%%%%%%%%%%%%%%%%%%%%%%%%%%%%%%%%%%%%%%%%%%%%%%%%%%%%%%%%%%%%%%%%%%%%%%%%%%%%%%%%%%%%%%%%%%%%%%%%%%%%%%%%%%%%%%%%%%%%%%%%%%
\begin{problem}{8 - (Page 15 - R. C. Hibbeler $14^{th}$ ed.)}
    Represent each of the following combinations of units
    in the correct SI form: (a) $\text{GN} \cdot \mu \text{m}$, (b) $\text{kg}/\mu \text{m}$, (c) $\text{N/ks}^2$,
    and (d) $\text{kN}/\mu \text{s}$.
\end{problem}
\begin{solution}
(a)
\[\text{GN} \cdot \mu \text{m}=10^9\text{ N}\cdot 10^{-6} \text{ m}=10^{9-6}\text{ N}\cdot \text{ m}=\]
\[10^3\text{ N}\cdot \text{ m}=\boxed{\text{kN}\cdot \text{m}}\]
(b)
\[\text{kg}/\mu \text{m}=\cfrac{10^3\text{ g}}{10^{-6}\text{ m}}=\cfrac{10^{3+6}\text{ g}}{m}=\]
\[\cfrac{10^9\text{ g}}{m}=\boxed{\cfrac{\text{Gg}}{m}}\]
(c)
\[\text{N/ks}^2=\cfrac{\text{N}}{(\text{10³ s})^2}=\cfrac{\text{N}}{10^{3\cdot 2}\text{s²}}=\]
\[\cfrac{10^{-6}\text{N}}{\text{ s²}}=\boxed{\cfrac{\mu\text{N}}{\text{s²}}}\]
(d)
\[\text{kN}/\mu \text{s}=\cfrac{10^3 \text{ N}}{10^{-6}\text{ s}}=\cfrac{10^{3+6}\text{ N}}{s}=\]
\[\cfrac{10^9 \text{ N}}{s}=\boxed{\cfrac{\text{GN}}{s}}\]
\end{solution}
%%%%%%%%%%%%%%%%%%%%%%%%%%%%%%%%%%%%%%%%%%%%%%%%%%%%%%%%%%%%%%%%%%%%%%%%%
% Problem 3
%%%%%%%%%%%%%%%%%%%%%%%%%%%%%%%%%%%%%%%%%%%%%%%%%%%%%%%%%%%%%%%%%%%%%%%%%%%%%%%%%%%%%%%%%%%%%%%%%%%%%%%%%%%%%%%%%%%%%%%%%%%%%%%%%%%%%%%%
\begin{problem}{14 - (Page 15 - R. C. Hibbeler $14^{th}$ ed.)}
    Evaluate each of the following to three significant
    figures and express each answer in SI units using an
    appropriate prefix: (a) 354 mg (45 km)/(0.0356 kN),
    (b) (0.004 53 Mg) (201 ms), (c) 435 MN/23.2 mm.
\end{problem}
\begin{solution}
    (a)
    \[354 \text{ mg}\cdot \cfrac{45 \text{ km}}{0.035\,6 \text{ kN}}=\cfrac{354 \times 10^{-3} \text{ g}\cdot 45\times 10^3\text{ m}}{0.035\, 6 \times 10^3 \text{ N}}=\cfrac{354 \times 45 \times \cancel{10^3}\times \cancel{10^{-3}}}{35.6}\cdot\left(\cfrac{\text{g}\cdot \text{m}}{\text{N}}\right)=\]
    \[\cfrac{44\underline{7}.471\,910\,112\text{ g}\cdot \text{m}}{N}=\boxed{\cfrac{44\underline{7}\text{ g}\cdot \text{m}}{\text{N}}}\]
    (b)
    \[0.004\,53 \text{ Mg}\cdot 201 \text{ ms}=0.004\, 53 \times 10^6 \text{ g} \cdot 201 \times 10^{-3} \text{s}=4.53\times \cancel{10^3} \times 201 \times \cancel{10^{-3}}\text{ g}\cdot \text{s}\]
    \[91\underline{0}.53\text{ g}\cdot \text{s}=\boxed{0.91\underline{0} \text{ kg}\cdot \text{s}}\]
    (c)
    \[\cfrac{435 \text{ MN}}{23.2 \text{ mm}}=\cfrac{435 \times 10^6 \text{ N}}{23.2 \times 10^{-3}\text{ m}}=\cfrac{435 \times 10^{6+3}\text{ N}}{23.2 \text{ m}}\]
    \[\cfrac{435 \text{ GN}}{23.2 \text{ m}}=\cfrac{18.\underline{7}5\text{ GN}}{\text{ m}}=\boxed{\cfrac{18.\underline{8} \text{ GN}}{\text{m}}} \]
\end{solution}
%%%%%%%%%%%%%%%%%%%%%%%%%%%%%%%%%%%%%%%%%%%%%%%%%%%%%%%%%%%%%%%%%%%%%%%%%
% Problem 4
%%%%%%%%%%%%%%%%%%%%%%%%%%%%%%%%%%%%%%%%%%%%%%%%%%%%%%%%%%%%%%%%%%%%%%%%%%%%%%%%%%%%%%%%%%%%%%%%%%%%%%%%%%%%%%%%%%%%%%%%%%%%%%%%%%%%%%%%
\begin{problem}{17 - (Page 15 - R. C. Hibbeler $14^{th}$ ed.)}
    A concrete column has a diameter of 350 mm and
    a length of 2 m. If the density (mass/volume) of concrete is
    $2.45\text{ Mg/m}^3$, determine the weight of the column.
\end{problem}
\begin{solution}
    For this problem we use the volume of the column times the density of concrete to find the answer, we just need to make sure our units are the same.
\[350 \text{ mm}\times \cfrac{1\text{ m}}{10^3 \text{ mm}}=\cfrac{350 \cancel{\text{ mm}} \times 1\text{ m}}{10^3 \cancel{\text{ mm}}}=350 \times 10^{-3} \text{ m}\]
\[\phi =  0.350 \text{ m}\]
The volume of a cylinder is given by the formula:
\[V=\pi r^2 h\]
replacing the values,
\[V=\pi \cdot \left(\cfrac{0.350}{2}\right)^2 \cdot 2=2\cdot 0.030625\pi=0.19\underline{2}\,422\,550\,032\]
\[0.192\text{ m³}\]
\[\cfrac{2.45 \times 10^6 \text{ g}}{\text{m³}} \cdot 0.192 \text{ m³}=\cfrac{2.45 \times 10^6 \text{ g}\cdot 0.192 \cancel{\text{ m³}}}{\cancel{\text{m³}}} =47\underline{0}\,400\text{ g}\]
kg is a mass unit measure, so we have to convert it to newtons (N) which is the correct unit for weight, and it is give by the formula:
\[W=m\cdot g\]
where $g$ is gravity and its value for our calculations is 9.81 m/s².
\[W=470.4 \text{ kg} \cdot 9.81 = 4\, 6\underline{1}4.624=\boxed{4.62 \text{ kN}}\]
\end{solution}
%%%%%%%%%%%%%%%%%%%%%%%%%%%%%%%%%%%%%%%%%%%%%%%%%%%%%%%%%%%%%%%%%%%%%%%%%
% Problem 5
%%%%%%%%%%%%%%%%%%%%%%%%%%%%%%%%%%%%%%%%%%%%%%%%%%%%%%%%%%%%%%%%%%%%%%%%%%%%%%%%%%%%%%%%%%%%%%%%%%%%%%%%%%%%%%%%%%%%%%%%%%%%%%%%%%%%%%%%
\begin{problem}{20 - (Page 15 - R. C. Hibbeler $14^{th}$ ed.)}
    Evaluate each of the following to three significant
figures and express each answer in SI units using an
appropriate prefix: (a) $(200 \text{ kN})^2$, (b) $(0.005\text{ mm})^2$, and (c)
$(400\text{ m})^3$.
\end{problem}
\begin{solution}
 (a)
 \[(200 \text{ kN})^2=(200 \times 10^3 \times \text{N})^2=(200)^2\times 10^{3\times 2}\times (\text{N})^2\]
 \[40\, 000 \times 10^6 \text{ N²} =40 \times 10^9 \text{ N²}=\boxed{ 40.0 \text{ GN²}}\]
 (b)
 \[(0.005\text{ mm})^2=(0.005)^2 \times (10^{-3})^2 \times (\text{m})^2 =0.000\,025 \times 10^{-6} \times \text{m²}\]
 \[0.025 \times 10^{-9}\times\text{m²}=\boxed{0.025\,0 \text{ nm²}}\text{ or } \boxed{25.0 \text{ pm²}}\]
 (c)
 \[(400\text{ m})^3=(400)^3\times(m)^3=64\,000\,000\text{ m³}\]
 \[64 \times 10^6 \text{ m³}=\boxed{ 64.0 \text{ Mm³}}\]
\end{solution}
%%%%%%%%%%%%%%%%%%%%%%%%%%%%%%%%%%%%%%%%%%%%%%%%%%%%%%%%%%%%%%%%%%%%%%%%%
\end{document}
